\newpage
\section{Implementation}

The Orthogonal Vector Visualization System is designed to be flexible, configurable, and easy to use. This section describes the implementation details of the system.

\subsection{Modular Architecture}

The system is organized into the following modules:

\begin{itemize}
    \item \texttt{vector\_utils.py}: Vector generation and component calculation functions.
    \item \texttt{visualization.py}: Comprehensive visualization functions for 2D and 3D plotting.
    \item \texttt{config.py}: Configuration management and serialization.
    \item \texttt{main.py}: Command-line interface and main program logic.
    \item \texttt{example\_circle.py}: Example generating a sphere-like pattern using orthogonal vectors.
    \item \texttt{example\_circle\_xy.py}: Example generating a traditional circle in the XY plane.
    \item \texttt{example\_orthogonal\_circle.py}: Example with improved visualization of orthogonal vectors.
\end{itemize}

This modular architecture allows for easy maintenance, extension, and reuse of the code.

\subsection{Vector Generation}

The \texttt{vector\_utils.py} module provides functions for vector generation and component calculation, including:

\begin{itemize}
    \item \texttt{create\_orthogonal\_vectors}: Generates vectors orthogonal to the x=y=z line from a given origin point, with options for standard or perfect circle generation.
    \item \texttt{create\_perfect\_orthogonal\_vectors}: Generates a single vector using the perfect orthogonal circle method.
    \item \texttt{create\_perfect\_orthogonal\_circle}: Generates multiple vectors forming a perfect circle in the plane orthogonal to the x=y=z line.
    \item \texttt{check\_orthogonality}: Checks if the component vectors are orthogonal.
    \item \texttt{calculate\_displacement\_vectors}: Calculates the displacement vectors from the origin.
    \item \texttt{calculate\_dot\_products}: Calculates the dot products between vectors.
\end{itemize}

The \texttt{create\_orthogonal\_vectors} function has been enhanced to support both the original scalar formula method and the new perfect circle generation method. When the \texttt{perfect} parameter is set to \texttt{True}, it uses the \texttt{create\_perfect\_orthogonal\_circle} function to generate a perfect circle in the plane orthogonal to the x=y=z line.

\subsection{Perfect Orthogonal Circle Implementation}

The perfect orthogonal circle implementation uses normalized basis vectors to ensure that all points are exactly at the specified distance from the origin and perfectly orthogonal to the (1,1,1) direction. The implementation includes:

\begin{itemize}
    \item \texttt{create\_perfect\_orthogonal\_vectors}: Generates a single vector at a specific angle using normalized basis vectors.
    \item \texttt{create\_perfect\_orthogonal\_circle}: Generates multiple vectors forming a perfect circle or circle segment by varying the angle parameter.
\end{itemize}

The perfect orthogonal circle method has several advantages over the original method:

\begin{itemize}
    \item \textbf{Exact Distance}: All points are exactly at the specified distance from the origin, with zero numerical error.
    \item \textbf{Perfect Orthogonality}: All points are perfectly orthogonal to the (1,1,1) direction, with dot products effectively zero.
    \item \textbf{Circle Segments}: The method supports generating partial circles by specifying start and end angles.
    \item \textbf{Arbitrary Origin}: The method supports an arbitrary origin point, not just the origin at (0,0,0).
\end{itemize}

The implementation has been thoroughly tested and verified to ensure that it meets these criteria.

\subsection{Visualization}

The \texttt{visualization.py} module provides comprehensive functions for visualizing vectors, including:

\begin{itemize}
    \item \texttt{plot\_vectors\_3d}: Plots vectors in 3D space with enhanced axis representation.
    \item \texttt{plot\_vectors\_2d\_projection}: Plots 2D projections (xy, xz, yz, r0 planes).
    \item \texttt{plot\_all\_projections}: Plots all projections of a single vector.
    \item \texttt{plot\_multiple\_vectors\_3d}: Plots multiple vectors in 3D with optional endpoints-only mode.
    \item \texttt{plot\_multiple\_vectors\_2d}: Plots 2D projections of multiple vectors.
    \item \texttt{plot\_multiple\_vectors}: Plots multiple vectors in all projections.
\end{itemize}

The visualization functions use Matplotlib to create the plots. The 3D visualization shows the vectors in three-dimensional space, while the 2D visualizations show projections onto the XY, XZ, and YZ planes. The endpoints-only option allows for clearer visualization of point patterns by only plotting the endpoints of vectors rather than the full arrows.

\subsubsection{Enhanced Visualization Features}

The visualization module has been enhanced with several features to improve clarity and spatial understanding:

\begin{itemize}
    \item \textbf{Color-coded Axes}: The X (red), Y (green), and Z (blue) axes are color-coded for easy identification.
    \item \textbf{Coordinate Labels}: Integer coordinate values are displayed along each axis, color-matched to the axis color.
    \item \textbf{Tick Marks}: Small tick marks along each axis for better spatial reference.
    \item \textbf{Data-driven Scaling}: The axis limits are dynamically adjusted based on the actual data points.
    \item \textbf{Equal Aspect Ratio}: The 3D plots maintain an equal aspect ratio for accurate spatial representation.
    \item \textbf{Buffer Zones}: Small buffer zones are added around the data points for better visibility.
\end{itemize}

These enhancements significantly improve the visual representation of the orthogonal vectors, making it easier to understand their spatial relationships and properties. The implementation includes careful consideration of color choices, label placement, and scaling to ensure optimal visualization regardless of the specific vector configuration being displayed.

\subsection{Configuration Management}

The \texttt{config.py} module implements a configuration management system with the \texttt{VectorConfig} class. This class provides a unified way to configure all aspects of vector generation and visualization, including:

\begin{itemize}
    \item Origin vector (R\_0)
    \item Distance parameter (d)
    \item Angle parameter (theta)
    \item Distance and angle ranges for multiple vector generation
    \item Endpoints-only plotting option
    \item Plot title and labels
    \item Plot saving options
\end{itemize}

The \texttt{VectorConfig} class also provides methods for saving configurations to and loading configurations from JSON files, making it easy to reuse configurations across different runs.

\subsection{Command-line Interface}

The \texttt{main.py} module provides a comprehensive command-line interface for the system. This interface allows users to generate and visualize orthogonal vectors with extensive options, including:

\begin{itemize}
    \item Setting the origin vector with -R or --origin
    \item Setting the distance parameter with -d or --distance
    \item Setting the angle parameter with -a or --angle
    \item Generating multiple vectors with --d-range and --theta-range
    \item Enabling endpoints-only plotting with --endpoints
    \item Controlling visualization options with --no-r0-plane, --no-legend, and --no-grid
    \item Saving plots with --save-plots and --output-dir
    \item Managing configurations with --config and --save-config
\end{itemize}

\subsection{Circle Examples}

The system includes three example scripts demonstrating different approaches to generating and visualizing circle and sphere-like patterns:

\begin{itemize}
    \item \texttt{example\_circle.py}: Generates points using orthogonal vector formulas, creating a sphere-like pattern.
    \item \texttt{example\_circle\_xy.py}: Creates a traditional circle in the XY plane.
    \item \texttt{example\_orthogonal\_circle.py}: Similar to the first example but with improved visualization.
\end{itemize}

These examples generate 73 points (from 0 to 360 degrees in 5-degree increments) and plot only the endpoints of the vectors, providing a clear visualization of the patterns formed.

\subsection{Dependencies}

The system depends on the following Python libraries:

\begin{itemize}
    \item \texttt{numpy}: For numerical computations
    \item \texttt{matplotlib}: For visualization
    \item \texttt{json}: For configuration file handling
    \item \texttt{argparse}: For command-line interface
\end{itemize}

These dependencies are specified in the \texttt{requirements.txt} file, making it easy to install them using pip and a virtual environment.
