\section{Implementation}

The generalized implementation of the Orthogonal Vectors Generator and Visualizer is designed to be modular, configurable, and easy to use. This section describes the implementation details of the package.

\subsection{Modular Architecture}

The package is organized into the following modules:

\begin{itemize}
    \item \texttt{vector\_utils.py}: Contains functions for vector calculations.
    \item \texttt{visualization.py}: Provides functions for visualizing vectors.
    \item \texttt{config.py}: Implements a configuration management system.
    \item \texttt{main.py}: Provides a command-line interface.
    \item \texttt{example.py}: Contains example scripts.
    \item \texttt{\_\_init\_\_.py}: Package initialization and exports.
\end{itemize}

This modular architecture allows for easy maintenance, extension, and reuse of the code.

\subsection{Vector Utilities}

The \texttt{vector\_utils.py} module provides functions for vector calculations, including:

\begin{itemize}
    \item \texttt{create\_orthogonal\_vectors}: Creates three orthogonal vectors from a given origin point.
    \item \texttt{check\_orthogonality}: Checks if a set of vectors is orthogonal.
    \item \texttt{calculate\_displacement\_vectors}: Calculates the displacement vectors from the origin.
    \item \texttt{calculate\_dot\_products}: Calculates the dot products between vectors.
\end{itemize}

The \texttt{create\_orthogonal\_vectors} function implements the mathematical formulation described in the previous section. It takes the origin point, distance parameter, and angle parameter as inputs and returns the three orthogonal vectors.

\subsection{Visualization}

The \texttt{visualization.py} module provides functions for visualizing vectors, including:

\begin{itemize}
    \item \texttt{plot\_vectors\_3d}: Plots vectors in 3D space.
    \item \texttt{plot\_vectors\_2d}: Plots vectors in various 2D projections.
    \item \texttt{plot\_vectors}: A high-level function that plots vectors in either 3D or 2D, depending on the configuration.
\end{itemize}

The visualization functions use Matplotlib to create the plots. The 3D visualization shows the vectors in three-dimensional space, while the 2D visualizations show projections onto the XY, XZ, and YZ planes, as well as a projection onto the plane containing the origin point.

\subsection{Configuration Management}

The \texttt{config.py} module implements a configuration management system with the \texttt{VectorConfig} class. This class provides a unified way to configure all aspects of vector generation and visualization, including:

\begin{itemize}
    \item Origin point
    \item Distance parameter
    \item Angle parameter
    \item Visualization type (3D or 2D)
    \item Plot title and labels
    \item Plot saving options
\end{itemize}

The \texttt{VectorConfig} class also provides methods for saving configurations to and loading configurations from JSON files, making it easy to reuse configurations across different runs.

\subsection{Command-line Interface}

The \texttt{main.py} module provides a command-line interface for the package. This interface allows users to generate and visualize orthogonal vectors without writing Python code. The command-line interface supports all the configuration options provided by the \texttt{VectorConfig} class, as well as additional options for controlling the behavior of the program.

\subsection{Package Initialization}

The \texttt{\_\_init\_\_.py} module initializes the package and exports the key functions and classes, making them available when the package is imported. This makes it easy to use the package in other Python projects.

\subsection{Example Scripts}

The \texttt{example.py} module contains example scripts demonstrating different use cases of the package. These examples serve as a starting point for users who want to use the package in their own projects.

\subsection{Dependencies}

The package depends on the following Python libraries:

\begin{itemize}
    \item \texttt{numpy}: For numerical computations
    \item \texttt{matplotlib}: For visualization
    \item \texttt{json}: For configuration file handling
    \item \texttt{argparse}: For command-line interface
\end{itemize}

These dependencies are specified in the \texttt{requirements.txt} file, making it easy to install them using pip.
