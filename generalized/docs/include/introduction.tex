\section{Introduction}

In three-dimensional space, orthogonal vectors are perpendicular to each other, meaning their dot product equals zero. This document describes a generalized implementation for generating and visualizing three orthogonal vectors from a given origin point.

\subsection{Project Overview}

The Generalized Orthogonal Vectors Generator and Visualizer is a Python package that provides a modular and configurable approach to vector generation and visualization. It is designed to be flexible, extensible, and easy to use, both as a command-line tool and as a Python package.

\subsection{Key Features}

The generalized implementation offers the following key features:

\begin{itemize}
    \item \textbf{Modular Architecture}: The implementation is divided into separate modules for vector calculations, visualization, and configuration management.
    
    \item \textbf{Configurability}: All aspects of vector generation and visualization can be configured through a unified configuration system.
    
    \item \textbf{Command-line Interface}: A comprehensive command-line interface allows for easy use without writing Python code.
    
    \item \textbf{Configuration File Support}: Configurations can be saved to and loaded from JSON files.
    
    \item \textbf{Multiple Visualization Options}: Both 3D visualization and various 2D projections are supported.
    
    \item \textbf{Plot Saving}: Plots can be saved to files instead of being displayed interactively.
    
    \item \textbf{Python Package}: The implementation can be used as a Python package, allowing for integration into other projects.
\end{itemize}

\subsection{Package Structure}

The package is organized into the following modules:

\begin{itemize}
    \item \texttt{vector\_utils.py}: Contains functions for vector calculations, including the creation of orthogonal vectors and checking their orthogonality.
    
    \item \texttt{visualization.py}: Provides functions for visualizing vectors in 3D and 2D projections.
    
    \item \texttt{config.py}: Implements a configuration management system with the \texttt{VectorConfig} class.
    
    \item \texttt{main.py}: Provides a command-line interface for the package.
    
    \item \texttt{example.py}: Contains example scripts demonstrating different use cases.
    
    \item \texttt{\_\_init\_\_.py}: Package initialization and exports.
\end{itemize}

\subsection{Document Structure}

This document is organized as follows:

\begin{itemize}
    \item \textbf{Mathematical Formulation}: Explains the mathematical basis for generating orthogonal vectors.
    
    \item \textbf{Implementation}: Describes the implementation details of the package.
    
    \item \textbf{API Reference}: Provides a reference for the package's API.
    
    \item \textbf{Usage Examples}: Shows examples of how to use the package.
    
    \item \textbf{Visualization}: Explains the visualization techniques used.
    
    \item \textbf{Configuration}: Describes the configuration management system.
    
    \item \textbf{Command-line Interface}: Documents the command-line interface.
    
    \item \textbf{Example Results}: Shows example results for different configurations.
    
    \item \textbf{Conclusion}: Summarizes the document and discusses potential future work.
    
    \item \textbf{Appendix}: Contains additional information, including complete code listings.
\end{itemize}
