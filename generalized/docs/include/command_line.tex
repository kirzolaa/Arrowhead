\section{Command-line Interface}

The Generalized Orthogonal Vectors Generator and Visualizer package provides a command-line interface that allows users to generate and visualize orthogonal vectors without writing Python code. This section describes the command-line interface and its features.

\subsection{Basic Usage}

The command-line interface can be accessed by running the \texttt{main.py} module:

\begin{lstlisting}[language=bash]
python -m generalized.main
\end{lstlisting}

This command generates and visualizes orthogonal vectors with default parameters (origin at [0, 0, 0], d=1.0, theta=pi/4).

You can also access detailed help information by running:

\begin{lstlisting}[language=bash]
python -m generalized.main help
\end{lstlisting}

This will display comprehensive information about all available commands, parameters, and usage examples.

\subsection{Command-line Options}

The command-line interface provides various options for configuring vector generation and visualization:

\begin{itemize}
    \item \texttt{--origin X Y Z}, \texttt{-R X Y Z}: Specifies the origin point as three space-separated values. Default: 0 0 0.
    \item \texttt{--distance D}, \texttt{-d D}: Specifies the distance parameter. Default: 1.0.
    \item \texttt{--angle THETA}, \texttt{-a THETA}: Specifies the angle parameter in radians. Default: 0.7853981633974483 (pi/4).
    \item \texttt{--plot-type TYPE}: Specifies the type of plot, either "3d" or "2d". Default: "3d".
    \item \texttt{--title TITLE}: Specifies the title of the plot. Default: None (uses a default title based on the plot type).
    \item \texttt{--no-show}: Prevents the plot from being displayed interactively. Default: False (shows the plot).
    \item \texttt{--save-path PATH}: Specifies the path to save the plot. Default: None (doesn't save the plot).
    \item \texttt{--config FILE}: Specifies a configuration file to load. Default: None (uses command-line options).
    \item \texttt{--save-config FILE}: Specifies a file to save the configuration to. Default: None (doesn't save the configuration).
    \item \texttt{--check-orthogonality}: Checks if the generated vectors are orthogonal and prints the result. Default: False (doesn't check).
    \item \texttt{--verbose}: Enables verbose output, including vector coordinates and dot products. Default: False (minimal output).
    \item \texttt{--help}: Shows the help message and exits.
\end{itemize}

\subsection{Examples}

\subsubsection{Customizing Vector Generation}

\begin{lstlisting}[language=bash]
python -m generalized.main -R 1 1 1 -d 2.0 -a 1.047
\end{lstlisting}

This command generates and visualizes orthogonal vectors with custom parameters (origin at [1, 1, 1], d=2.0, theta=pi/3). Note the use of shorter flags (-R, -d, -a) for more concise commands.

\subsubsection{Customizing Visualization}

\begin{lstlisting}[language=bash]
python -m generalized.main --plot-type 2d --title "Custom Visualization" --save-path custom.png
\end{lstlisting}

This command generates orthogonal vectors with default parameters and visualizes them with custom visualization options (2D plot, custom title, save to file).

\subsubsection{Using a Configuration File}

\begin{lstlisting}[language=bash]
python -m generalized.main --config config.json
\end{lstlisting}

This command loads a configuration from a file and uses it to generate and visualize orthogonal vectors.

\subsubsection{Saving a Configuration File}

\begin{lstlisting}[language=bash]
python -m generalized.main --origin 1 1 1 --d 2.0 --theta 1.047 --save-config config.json
\end{lstlisting}

This command generates and visualizes orthogonal vectors with custom parameters and saves the configuration to a file.

\subsubsection{Checking Orthogonality}

\begin{lstlisting}[language=bash]
python -m generalized.main --check-orthogonality
\end{lstlisting}

This command generates orthogonal vectors with default parameters, visualizes them, and checks if they are orthogonal.

\subsubsection{Verbose Output}

\begin{lstlisting}[language=bash]
python -m generalized.main --verbose
\end{lstlisting}

This command generates orthogonal vectors with default parameters, visualizes them, and prints verbose output, including vector coordinates and dot products.

\subsection{Implementation Details}

The command-line interface is implemented in the \texttt{main.py} module using the \texttt{argparse} module from the Python standard library. The module defines a \texttt{main} function that parses command-line arguments, creates a configuration, generates orthogonal vectors, and visualizes them.

The command-line interface follows these steps:

\begin{enumerate}
    \item Parse command-line arguments using \texttt{argparse}.
    \item If a configuration file is specified, load the configuration from the file.
    \item Override the configuration with any command-line options that are specified.
    \item Generate orthogonal vectors using the configuration.
    \item If requested, check if the vectors are orthogonal and print the result.
    \item If verbose output is enabled, print vector coordinates and dot products.
    \item Visualize the vectors using the configuration.
    \item If requested, save the configuration to a file.
\end{enumerate}

\subsection{Error Handling}

The command-line interface includes error handling for various scenarios, including:

\begin{itemize}
    \item Invalid command-line arguments (e.g., non-numeric values for numeric options).
    \item Invalid configuration file (e.g., file not found, invalid JSON).
    \item Invalid configuration parameters (e.g., negative distance parameter).
\end{itemize}

When an error occurs, the command-line interface prints an error message and exits with a non-zero exit code.

\subsection{Help Message}

The command-line interface provides a help message that can be displayed using the \texttt{--help} option:

\begin{lstlisting}[language=bash]
python -m generalized.main --help
\end{lstlisting}

The help message includes a description of the program, a list of all available options, and examples of how to use the program.
