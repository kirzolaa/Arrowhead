\newpage
\section{Command-line Interface}

The Generalized Arrowhead Framework provides a comprehensive command-line interface that allows users to generate and visualize vectors orthogonal to the x=y=z line, as well as generate and analyze arrowhead matrices. This section describes the command-line interface and its features.

\subsection{Basic Usage}

The command-line interface can be accessed by running the \texttt{main.py} module with the appropriate subcommand:

\begin{lstlisting}[language=bash]
# For vector generation and visualization
python generalized/main.py vector

# For arrowhead matrix generation and analysis
python generalized/main.py arrowhead

# For detailed help information
python generalized/main.py help
\end{lstlisting}

The \texttt{vector} subcommand generates and visualizes a single vector with default parameters (origin at [0, 0, 0], d=1.0, theta=pi/4).

The \texttt{arrowhead} subcommand generates and analyzes arrowhead matrices with default parameters (4x4 matrix, 72 theta steps).

\subsection{Command-line Options}

The command-line interface provides extensive options for both vector generation and arrowhead matrix analysis. The options are organized by subcommand.

\subsubsection{Vector Generation Options}

The following options are available for the \texttt{vector} subcommand:

\begin{itemize}
    \item \texttt{-R X Y Z}, \texttt{--origin X Y Z}: Sets the origin vector R\_0 coordinates. Default: 0 0 0.
    \item \texttt{-d VALUE}, \texttt{--distance VALUE}: Sets the distance parameter. Default: 1.0.
    \item \texttt{--d-range START STEPS END}: Generates multiple vectors with distance values from START to END with STEPS steps.
    \item \texttt{-a VALUE}, \texttt{--angle VALUE}, \texttt{--theta VALUE}: Sets the angle parameter in radians. Default: 0.7853981633974483 (pi/4).
    \item \texttt{--theta-range START STEPS END}: Generates multiple vectors with angle values from START to END with STEPS steps.
    \item \texttt{--perfect}: Uses the perfect orthogonal circle method for vector generation.
    \item \texttt{--plot-type TYPE}: Specifies the type of plot, either "3d" or "2d". Default: "3d".
    \item \texttt{--title TITLE}: Specifies the title of the plot.
    \item \texttt{--no-show}: Prevents the plot from being displayed interactively.
    \item \texttt{--save-path PATH}: Specifies the path to save the plot.
    \item \texttt{--no-enhanced-visualization}: Disables enhanced visualization features.
    \item \texttt{--axis-colors X Y Z}: Sets custom colors for the X, Y, and Z axes. Default: 'r g b'.
    \item \texttt{--no-coordinate-labels}: Disables coordinate labels on the axes.
    \item \texttt{--no-equal-aspect-ratio}: Disables equal aspect ratio for 3D plots.
    \item \texttt{--buffer-factor VALUE}: Sets the buffer factor for axis limits. Default: 0.2.
    \item \texttt{--no-r0-plane}: Does not show the R\_0 plane projection.
    \item \texttt{--no-legend}: Does not show the legend.
    \item \texttt{--no-grid}: Does not show the grid.
    \item \texttt{--endpoints true/false}: Only plot the endpoints of vectors, not the arrows. Default: false.
    \item \texttt{--save-plots}: Save plots to files instead of displaying them.
    \item \texttt{--output-dir DIR}: Directory to save plots to. Default: 'plots'.
    \item \texttt{--config FILE}: Load configuration from a JSON file.
    \item \texttt{--save-config FILE}: Save current configuration to a JSON file.
\end{itemize}

\subsubsection{Arrowhead Matrix Options}

The following options are available for the \texttt{arrowhead} subcommand:

\begin{itemize}
    \item \texttt{--r0 X Y Z}: Origin vector coordinates. Default: 0 0 0.
    \item \texttt{--d VALUE}: Distance parameter. Default: 0.5.
    \item \texttt{--theta-start VALUE}: Starting theta value in radians. Default: 0.
    \item \texttt{--theta-end VALUE}: Ending theta value in radians. Default: $2\pi$.
    \item \texttt{--theta-steps VALUE}: Number of theta values to generate matrices for. Default: 72.
    \item \texttt{--coupling VALUE}: Coupling constant for off-diagonal elements. Default: 0.1.
    \item \texttt{--omega VALUE}: Angular frequency for the energy term $\hbar\omega$. Default: 1.0.
    \item \texttt{--size VALUE}: Size of the matrix to generate. Default: 4.
    \item \texttt{--perfect}: Use perfect circle generation method. Default: True.
    \item \texttt{--output-dir DIR}: Directory to save results. Default: './results'.
    \item \texttt{--load-only}: Only load existing results and create plots.
    \item \texttt{--plot-only}: Only create plots from existing results.
\end{itemize}



\subsection{Examples}

\subsubsection{Vector Generation Examples}

\paragraph{Generating a Single Vector}

\begin{lstlisting}[language=bash]
python generalized/main.py vector -R 1 1 1 -d 2.0 -a 1.047
\end{lstlisting}

This command generates and visualizes a single vector with custom parameters (origin at [1, 1, 1], d=2.0, theta=pi/3).

\paragraph{Generating Multiple Vectors with Distance Range}

\begin{lstlisting}[language=bash]
python generalized/main.py vector -R 0 0 0 --d-range 1 5 3 -a 0.7854
\end{lstlisting}

This command generates and visualizes multiple vectors with varying distance values (from 1 to 3 in 5 steps) and fixed angle (pi/4).

\paragraph{Generating Multiple Vectors with Angle Range}

\begin{lstlisting}[language=bash]
python generalized/main.py vector -R 0 0 0 -d 1.5 --theta-range 0 10 3.14159
\end{lstlisting}

This command generates and visualizes multiple vectors with fixed distance (1.5) and varying angle values (from 0 to pi in 10 steps).

\paragraph{Using 2D Plot Type}

\begin{lstlisting}[language=bash]
python generalized/main.py vector --plot-type 2d --title "2D Projection of Orthogonal Vector"
\end{lstlisting}

This command generates a vector with default parameters and displays it using a 2D plot with a custom title.

\paragraph{Customizing Visualization}

\begin{lstlisting}[language=bash]
python generalized/main.py vector --axis-colors blue green red --no-coordinate-labels
\end{lstlisting}

This command generates a vector with default parameters and customizes the visualization with custom axis colors and no coordinate labels.

\paragraph{Saving to a Specific Path}

\begin{lstlisting}[language=bash]
python generalized/main.py vector --no-show --save-path "./figures/orthogonal_vector.png"
\end{lstlisting}

\subsubsection{Arrowhead Matrix Examples}

\paragraph{Generating Matrices with Default Parameters}

\begin{lstlisting}[language=bash]
python generalized/main.py arrowhead
\end{lstlisting}

This command generates and analyzes arrowhead matrices with default parameters (4x4 matrix, 72 theta steps).

\paragraph{Generating Matrices with Custom Parameters}

\begin{lstlisting}[language=bash]
python generalized/main.py arrowhead --r0 1 1 1 --d 0.8 --theta-steps 36 --size 6
\end{lstlisting}

This command generates and analyzes 6x6 arrowhead matrices with custom parameters (origin at [1, 1, 1], d=0.8, 36 theta steps).

\paragraph{Using Perfect Circle Generation}

\begin{lstlisting}[language=bash]
python generalized/main.py arrowhead --perfect --theta-steps 12
\end{lstlisting}

This command generates and analyzes arrowhead matrices using the perfect circle generation method with 12 theta steps.

\paragraph{Only Creating Plots from Existing Results}

\begin{lstlisting}[language=bash]
python generalized/main.py arrowhead --plot-only --output-dir my_results
\end{lstlisting}

This command only creates plots from existing results in the specified directory.

\paragraph{Loading Existing Results and Creating Plots}

\begin{lstlisting}[language=bash]
python generalized/main.py arrowhead --load-only --output-dir my_results
\end{lstlisting}

This command loads existing results from the specified directory and creates plots.

\subsubsection{Endpoints-only Plotting}

\begin{lstlisting}[language=bash]
python generalized/main.py --endpoints true
\end{lstlisting}

This command generates a vector with default parameters and plots only the endpoints, not the arrows.

\subsubsection{Saving Plots}

\begin{lstlisting}[language=bash]
python generalized/main.py --save-plots --output-dir my_plots
\end{lstlisting}

This command generates a vector with default parameters and saves the plots to the 'my\_plots' directory.

\subsubsection{Using a Configuration File}

\begin{lstlisting}[language=bash]
python generalized/main.py --config my_config.json
\end{lstlisting}

This command loads a configuration from a file and uses it to generate and visualize vectors.

\subsubsection{Saving a Configuration File}

\begin{lstlisting}[language=bash]
python generalized/main.py -R 1 1 1 -d 2.0 -a 1.047 --save-config my_config.json
\end{lstlisting}

This command generates and visualizes a vector with custom parameters and saves the configuration to a file.

\subsection{Configuration File Format}

The configuration file is a JSON file with the following structure:

\begin{lstlisting}[language=json]
{
    "origin": [x, y, z],
    "d": float,
    "theta": float,
    "plot_type": "3d" or "2d",
    "title": string or null,
    "show_plot": boolean,
    "save_path": string or null,
    "enhanced_visualization": boolean,
    "axis_colors": ["r", "g", "b"],
    "show_coordinate_labels": boolean,
    "equal_aspect_ratio": boolean,
    "buffer_factor": float,
    "show_r0_plane": boolean,
    "show_legend": boolean,
    "show_grid": boolean,
    "perfect": boolean
}
\end{lstlisting}

All fields are optional and will use default values if not specified. The default configuration is as follows:

\begin{itemize}
    \item \texttt{origin}: [0, 0, 0]
    \item \texttt{d}: 1.0
    \item \texttt{theta}: $\pi/4$ (approximately 0.7853981633974483)
    \item \texttt{plot\_type}: "3d"
    \item \texttt{title}: null (uses a default title based on the plot type)
    \item \texttt{show\_plot}: true
    \item \texttt{save\_path}: null (doesn't save the plot)
    \item \texttt{enhanced\_visualization}: true
    \item \texttt{axis\_colors}: ["r", "g", "b"]
    \item \texttt{show\_coordinate\_labels}: true
    \item \texttt{equal\_aspect\_ratio}: true
    \item \texttt{buffer\_factor}: 0.2
    \item \texttt{show\_r0\_plane}: true
    \item \texttt{show\_legend}: true
    \item \texttt{show\_grid}: true
    \item \texttt{perfect}: false
\end{itemize}

\subsection{Circle Examples}

The system includes several example scripts for generating and visualizing circle and sphere-like patterns:

\begin{lstlisting}[language=bash]
# Generate a sphere-like pattern using vectors orthogonal to the x=y=z line
python generalized/example_circle.py

# Generate a traditional circle in the XY plane
python generalized/example_circle_xy.py

# Generate a sphere-like pattern with improved visualization
# using vectors orthogonal to the x=y=z line
python generalized/example_orthogonal_circle.py

# Generate a perfect circle in the plane orthogonal to the x=y=z line
# with enhanced visualization features
python perfect_orthogonal_circle.py

# Generate multiple perfect circles at different distances
# with enhanced visualization features
python perfect_circle_distance_range.py
\end{lstlisting}

These examples generate 73 points (from 0\textdegree\ to 360\textdegree\ in 5\textdegree\ increments) and plot only the endpoints of the vectors, providing a clear visualization of the patterns formed. The orthogonality to the x=y=z line is ensured by using the basis vectors [1, -1/2, -1/2] and [0, -1/2, 1/2] in the vector generation process.

\subsection{Implementation Details}

The command-line interface is implemented in the \texttt{main.py} module using the \texttt{argparse} module from the Python standard library. The module defines a \texttt{main} function that parses command-line arguments, creates a configuration, generates vectors, and visualizes them.

The command-line interface follows these steps:

\begin{enumerate}
    \item Parse command-line arguments using \texttt{argparse}.
    \item If a configuration file is specified, load the configuration from the file.
    \item Override the configuration with any command-line options that are specified.
    \item Generate a vector using the scalar-based formula and the configuration.
    \item If requested, analyze the properties of the generated vector and print the result.
    \item If verbose output is enabled, print vector coordinates and properties.
    \item Visualize the vectors using the configuration.
    \item If requested, save the configuration to a file.
\end{enumerate}

\subsection{Error Handling}

The command-line interface includes error handling for various scenarios, including:

\begin{itemize}
    \item Invalid command-line arguments (e.g., non-numeric values for numeric options).
    \item Invalid configuration file (e.g., file not found, invalid JSON).
    \item Invalid configuration parameters (e.g., negative distance parameter).
\end{itemize}

When an error occurs, the command-line interface prints an error message and exits with a non-zero exit code.

\subsection{Help Message}

The command-line interface provides a help message that can be displayed using the \texttt{--help} option:

\begin{lstlisting}[language=bash]
python -m generalized.main --help
\end{lstlisting}

The help message includes a description of the program, a list of all available options, and examples of how to use the program.
