\section{Usage Examples}

This section provides examples of how to use the Generalized Orthogonal Vectors Generator and Visualizer package. It includes examples of using the package as a Python module and as a command-line tool.

\subsection{Basic Usage as a Python Module}

The following example shows how to use the package as a Python module to generate and visualize orthogonal vectors with default parameters:

\begin{lstlisting}[language=Python]
import numpy as np
from generalized import create_orthogonal_vectors, plot_vectors

# Generate orthogonal vectors with default parameters
# (origin at [0, 0, 0], d=1.0, theta=pi/4)
vectors = create_orthogonal_vectors(origin=[0, 0, 0])

# Plot the vectors in 3D
plot_vectors(vectors, origin=[0, 0, 0])
\end{lstlisting}

\subsection{Customizing Vector Generation}

The following example shows how to customize the vector generation by specifying the origin, distance parameter, and angle parameter:

\begin{lstlisting}[language=Python]
import numpy as np
import math
from generalized import create_orthogonal_vectors, plot_vectors

# Generate orthogonal vectors with custom parameters
origin = [1, 1, 1]
d = 2.0
theta = math.pi / 3

vectors = create_orthogonal_vectors(origin=origin, d=d, theta=theta)

# Plot the vectors in 3D
plot_vectors(vectors, origin=origin, title=f"Orthogonal Vectors (Origin={origin}, d={d}, theta={theta})")
\end{lstlisting}

\subsection{Using the VectorConfig Class}

The following example shows how to use the \texttt{VectorConfig} class to configure vector generation and visualization:

\begin{lstlisting}[language=Python]
import numpy as np
import math
from generalized import create_orthogonal_vectors, plot_vectors, VectorConfig

# Create a configuration
config = VectorConfig(
    origin=[0, 0, 2],
    d=1.5,
    theta=math.pi / 6,
    plot_type="2d",
    title="Custom Configuration",
    save_path="custom_config.png"
)

# Generate orthogonal vectors using the configuration
vectors = create_orthogonal_vectors(
    origin=config.origin,
    d=config.d,
    theta=config.theta
)

# Plot the vectors using the configuration
plot_vectors(
    vectors,
    origin=config.origin,
    plot_type=config.plot_type,
    title=config.title,
    show_plot=config.show_plot,
    save_path=config.save_path
)
\end{lstlisting}

\subsection{Saving and Loading Configurations}

The following example shows how to save a configuration to a file and load it later:

\begin{lstlisting}[language=Python]
import numpy as np
import math
from generalized import VectorConfig, create_orthogonal_vectors, plot_vectors

# Create a configuration
config = VectorConfig(
    origin=[0, 0, 2],
    d=1.5,
    theta=math.pi / 6,
    plot_type="2d",
    title="Custom Configuration"
)

# Save the configuration to a file
config.save_to_file("config.json")

# Later, load the configuration from the file
loaded_config = VectorConfig.load_from_file("config.json")

# Generate orthogonal vectors using the loaded configuration
vectors = create_orthogonal_vectors(
    origin=loaded_config.origin,
    d=loaded_config.d,
    theta=loaded_config.theta
)

# Plot the vectors using the loaded configuration
plot_vectors(
    vectors,
    origin=loaded_config.origin,
    plot_type=loaded_config.plot_type,
    title=loaded_config.title,
    show_plot=loaded_config.show_plot,
    save_path=loaded_config.save_path
)
\end{lstlisting}

\subsection{Checking Orthogonality}

The following example shows how to check if a set of vectors is orthogonal:

\begin{lstlisting}[language=Python]
import numpy as np
from generalized import create_orthogonal_vectors, check_orthogonality

# Generate orthogonal vectors
vectors = create_orthogonal_vectors(origin=[0, 0, 0])

# Check if the vectors are orthogonal
is_orthogonal = check_orthogonality(vectors, origin=[0, 0, 0])

print(f"Vectors are orthogonal: {is_orthogonal}")
\end{lstlisting}

\subsection{Using the Command-line Interface}

The following examples show how to use the command-line interface to generate and visualize orthogonal vectors:

\subsubsection{Basic Usage}

\begin{lstlisting}[language=bash]
python -m generalized.main
\end{lstlisting}

This command generates and visualizes orthogonal vectors with default parameters (origin at [0, 0, 0], d=1.0, theta=pi/4).

\subsubsection{Customizing Vector Generation}

\begin{lstlisting}[language=bash]
python -m generalized.main --origin 1 1 1 --d 2.0 --theta 1.047
\end{lstlisting}

This command generates and visualizes orthogonal vectors with custom parameters (origin at [1, 1, 1], d=2.0, theta=pi/3).

\subsubsection{Customizing Visualization}

\begin{lstlisting}[language=bash]
python -m generalized.main --plot-type 2d --title "Custom Visualization" --save-path custom.png
\end{lstlisting}

This command generates orthogonal vectors with default parameters and visualizes them with custom visualization options (2D plot, custom title, save to file).

\subsubsection{Using a Configuration File}

\begin{lstlisting}[language=bash]
python -m generalized.main --config config.json
\end{lstlisting}

This command loads a configuration from a file and uses it to generate and visualize orthogonal vectors.

\subsubsection{Saving a Configuration File}

\begin{lstlisting}[language=bash]
python -m generalized.main --origin 1 1 1 --d 2.0 --theta 1.047 --save-config config.json
\end{lstlisting}

This command generates and visualizes orthogonal vectors with custom parameters and saves the configuration to a file.

\subsection{Complete Example Script}

The following is a complete example script that demonstrates various features of the package:

\begin{lstlisting}[language=Python]
import numpy as np
import math
import matplotlib.pyplot as plt
from generalized import create_orthogonal_vectors, check_orthogonality, plot_vectors, VectorConfig

def main():
    # Create configurations for different examples
    configs = [
        VectorConfig(
            origin=[0, 0, 0],
            d=1.0,
            theta=math.pi / 4,
            plot_type="3d",
            title="Default Configuration",
            save_path="default.png"
        ),
        VectorConfig(
            origin=[1, 1, 1],
            d=2.0,
            theta=math.pi / 3,
            plot_type="3d",
            title="Custom Configuration 1",
            save_path="custom1.png"
        ),
        VectorConfig(
            origin=[0, 0, 2],
            d=1.5,
            theta=math.pi / 6,
            plot_type="2d",
            title="Custom Configuration 2",
            save_path="custom2.png"
        )
    ]
    
    # Process each configuration
    for i, config in enumerate(configs):
        print(f"Processing configuration {i+1}/{len(configs)}")
        
        # Generate orthogonal vectors
        vectors = create_orthogonal_vectors(
            origin=config.origin,
            d=config.d,
            theta=config.theta
        )
        
        # Check orthogonality
        is_orthogonal = check_orthogonality(vectors, origin=config.origin)
        print(f"  Vectors are orthogonal: {is_orthogonal}")
        
        # Plot vectors
        plot_vectors(
            vectors,
            origin=config.origin,
            plot_type=config.plot_type,
            title=config.title,
            show_plot=False,
            save_path=config.save_path
        )
        print(f"  Plot saved to {config.save_path}")
        
        # Save configuration
        config_file = f"config{i+1}.json"
        config.save_to_file(config_file)
        print(f"  Configuration saved to {config_file}")
    
    # Show all plots
    plt.show()

if __name__ == "__main__":
    main()
\end{lstlisting}

This script creates three different configurations, generates orthogonal vectors for each, checks their orthogonality, plots them, and saves both the plots and configurations to files. Finally, it displays all the plots.
