\section{Source Code}

This appendix contains the complete source code for the Generalized Orthogonal Vectors Generator and Visualizer package.

\subsection{main.py}

\begin{lstlisting}[language=Python]
#!/usr/bin/env python3
import numpy as np
import matplotlib.pyplot as plt
import argparse
import math
import os
import sys

from vector_utils import create_orthogonal_vectors, check_orthogonality
from visualization import plot_vectors_3d, plot_vectors_2d_projection, plot_all_projections
from config import VectorConfig, default_config

def parse_arguments():
    """
    Parse command line arguments
    
    Returns:
    argparse.Namespace: Parsed arguments
    """
    parser = argparse.ArgumentParser(description='Generate and visualize orthogonal vectors')
    
    # Vector parameters
    parser.add_argument('--origin', '-R', type=float, nargs=3, default=[0, 0, 0],
                        help='Origin vector R_0 (x y z)')
    parser.add_argument('--distance', '-d', type=float, default=1,
                        help='Distance parameter d')
    parser.add_argument('--angle', '-a', type=float, default=math.pi/4,
                        help='Angle parameter theta in radians')
    
    # Visualization parameters
    parser.add_argument('--no-r0-plane', action='store_false', dest='show_r0_plane',
                        help='Do not show the R_0 plane projection')
    parser.add_argument('--no-legend', action='store_false', dest='show_legend',
                        help='Do not show the legend')
    parser.add_argument('--no-grid', action='store_false', dest='show_grid',
                        help='Do not show the grid')
    
    # Output parameters
    parser.add_argument('--save-plots', action='store_true',
                        help='Save plots to files instead of displaying them')
    parser.add_argument('--output-dir', type=str, default='plots',
                        help='Directory to save plots to')
    parser.add_argument('--config', type=str,
                        help='Path to configuration file')
    parser.add_argument('--save-config', type=str,
                        help='Save configuration to file')
    
    return parser.parse_args()

def display_help():
    """
    Display detailed help information
    """
    help_text = """
    Orthogonal Vectors Generator and Visualizer
    =======================================
    
    This tool generates and visualizes three orthogonal vectors from a given origin point.
    
    Basic Usage:
    -----------
    python main.py                              # Use default parameters
    python main.py -R 1 1 1                    # Set origin to (1,1,1)
    python main.py -R 0 0 2 -d 1.5 -a 0.5236   # Custom origin, distance and angle
    python main.py --help                      # Show help
    
    Parameters:
    ----------
    -R, --origin X Y Z    : Set the origin vector R_0 coordinates (default: 0 0 0)
    -d, --distance VALUE  : Set the distance parameter (default: 1)
    -a, --angle VALUE     : Set the angle parameter in radians (default: \pi/4)
    
    Visualization Options:
    --------------------
    --no-r0-plane        : Do not show the R_0 plane projection
    --no-legend          : Do not show the legend
    --no-grid            : Do not show the grid
    
    Output Options:
    --------------
    --save-plots         : Save plots to files instead of displaying them
    --output-dir DIR     : Directory to save plots to (default: 'plots')
    
    Configuration:
    -------------
    --config FILE        : Load configuration from a JSON file
    --save-config FILE   : Save current configuration to a JSON file
    
    Examples:
    --------
    # Generate vectors with origin at (1,1,1), distance 2, and angle \pi/3
    python main.py -R 1 1 1 -d 2 -a 1.047
    
    # Save plots to a custom directory
    python main.py -R 0 0 2 --save-plots --output-dir my_plots
    
    # Load configuration from a file
    python main.py --config my_config.json
    """
    print(help_text)
    sys.exit(0)

def main():
    """
    Main function
    """
    # Check for detailed help command
    if len(sys.argv) > 1 and sys.argv[1] == 'help':
        display_help()
    
    # Parse command line arguments
    args = parse_arguments()
    
    # Load configuration
    if args.config:
        config = VectorConfig.load_from_file(args.config)
    else:
        # Create configuration from command line arguments
        config = VectorConfig(
            R_0=args.origin,
            d=args.distance,
            theta=args.angle,
            show_r0_plane=args.show_r0_plane,
            show_legend=args.show_legend,
            show_grid=args.show_grid
        )
    
    # Save configuration if requested
    if args.save_config:
        config.save_to_file(args.save_config)
    
    # Create the orthogonal vectors
    R_0 = config.R_0
    R_1, R_2, R_3 = create_orthogonal_vectors(R_0, config.d, config.theta)
    
    # Print vector information
    print("R_0:", R_0)
    print("R_1:", R_1)
    print("R_2:", R_2)
    print("R_3:", R_3)
    
    # Check orthogonality
    orthogonality = check_orthogonality(R_0, R_1, R_2, R_3)
    print("\nChecking orthogonality (dot products should be close to zero):")
    for key, value in orthogonality.items():
        print(f"{key}: {value}")
    
    # Plot the vectors
    plots = plot_all_projections(
        R_0, R_1, R_2, R_3,
        show_r0_plane=config.show_r0_plane,
        figsize_3d=config.figsize_3d,
        figsize_2d=config.figsize_2d
    )
    
    # Save or show the plots
    if args.save_plots:
        # Create output directory if it doesn't exist
        os.makedirs(args.output_dir, exist_ok=True)
        
        # Save each plot
        for name, (fig, _) in plots.items():
            filename = os.path.join(args.output_dir, f"{name}.png")
            fig.savefig(filename)
            print(f"Saved plot to {filename}")
    else:
        # Show the plots
        plt.show()

if __name__ == "__main__":
    main()
\end{lstlisting}

\subsection{vector\_utils.py}

\begin{lstlisting}[language=Python]
#!/usr/bin/env python3
import numpy as np

def create_orthogonal_vectors(R_0=(0, 0, 0), d=1, theta=0):
    """
    Create 3 orthogonal R vectors for R_0
    
    Parameters:
    R_0 (tuple or numpy.ndarray): The origin vector, default is (0, 0, 0)
    d (float): The distance parameter, default is 1
    theta (float): The angle parameter in radians, default is 0
    
    Returns:
    tuple: Three orthogonal vectors R_1, R_2, R_3
    """
    # Convert R_0 to numpy array for vector operations
    R_0 = np.array(R_0)
    
    # Calculate R_1, R_2, R_3 according to the given formulas
    # R_1 = R_0 + d * (cos(theta))*sqrt(2/3)
    R_1 = R_0 + d * np.cos(theta) * np.sqrt(2/3) * np.array([1, -1/2, -1/2])
    
    # R_2 = R_0 + d * (cos(theta)/sqrt(3) + sin(theta))/sqrt(2)
    R_2 = R_0 + d * (np.cos(theta)/np.sqrt(3) + np.sin(theta))/np.sqrt(2) * np.array([1, 1, 1])
    
    # R_3 = R_0 + d * (sin(theta) - cos(theta)/sqrt(3))/sqrt(2)
    R_3 = R_0 + d * (np.sin(theta) - np.cos(theta)/np.sqrt(3))/np.sqrt(2) * np.array([0, -1/2, 1/2]) * np.sqrt(2)
    
    return R_1, R_2, R_3

def check_orthogonality(R_0, R_1, R_2, R_3):
    """
    Check if the vectors R_1, R_2, R_3 are orthogonal with respect to R_0
    
    Parameters:
    R_0, R_1, R_2, R_3 (numpy.ndarray): The vectors to check
    
    Returns:
    dict: Dictionary containing the dot products between pairs of vectors
    """
    dot_1_2 = np.dot(R_1 - R_0, R_2 - R_0)
    dot_1_3 = np.dot(R_1 - R_0, R_3 - R_0)
    dot_2_3 = np.dot(R_2 - R_0, R_3 - R_0)
    
    return {
        "R_1 $\cdot$ R_2": dot_1_2,
        "R_1 $\cdot$ R_3": dot_1_3,
        "R_2 $\cdot$ R_3": dot_2_3
    }
\end{lstlisting}

\subsection{config.py}

\begin{lstlisting}[language=Python]
#!/usr/bin/env python3
import numpy as np
import math
import json
import os

class VectorConfig:
    """
    Configuration class for orthogonal vector generation and visualization
    """
    def __init__(self, 
                 R_0=(0, 0, 0), 
                 d=1, 
                 theta=math.pi/4,
                 show_r0_plane=True,
                 figsize_3d=(10, 8),
                 figsize_2d=(8, 8),
                 show_legend=True,
                 show_grid=True):
        """
        Initialize the configuration
        
        Parameters:
        R_0 (tuple or list): The origin vector
        d (float): The distance parameter
        theta (float): The angle parameter in radians
        show_r0_plane (bool): Whether to show the R_0 plane projection
        figsize_3d (tuple): Figure size for 3D plot
        figsize_2d (tuple): Figure size for 2D plots
        show_legend (bool): Whether to show the legend
        show_grid (bool): Whether to show the grid
        """
        self.R_0 = np.array(R_0)
        self.d = d
        self.theta = theta
        self.show_r0_plane = show_r0_plane
        self.figsize_3d = figsize_3d
        self.figsize_2d = figsize_2d
        self.show_legend = show_legend
        self.show_grid = show_grid
    
    def to_dict(self):
        """
        Convert the configuration to a dictionary
        
        Returns:
        dict: Dictionary representation of the configuration
        """
        return {
            'R_0': self.R_0.tolist(),
            'd': self.d,
            'theta': self.theta,
            'show_r0_plane': self.show_r0_plane,
            'figsize_3d': self.figsize_3d,
            'figsize_2d': self.figsize_2d,
            'show_legend': self.show_legend,
            'show_grid': self.show_grid
        }
    
    @classmethod
    def from_dict(cls, config_dict):
        """
        Create a configuration from a dictionary
        
        Parameters:
        config_dict (dict): Dictionary containing configuration parameters
        
        Returns:
        VectorConfig: Configuration object
        """
        return cls(
            R_0=config_dict.get('R_0', (0, 0, 0)),
            d=config_dict.get('d', 1),
            theta=config_dict.get('theta', math.pi/4),
            show_r0_plane=config_dict.get('show_r0_plane', True),
            figsize_3d=config_dict.get('figsize_3d', (10, 8)),
            figsize_2d=config_dict.get('figsize_2d', (8, 8)),
            show_legend=config_dict.get('show_legend', True),
            show_grid=config_dict.get('show_grid', True)
        )
    
    def save_to_file(self, filename):
        """
        Save the configuration to a JSON file
        
        Parameters:
        filename (str): Path to the output file
        """
        with open(filename, 'w') as f:
            json.dump(self.to_dict(), f, indent=4)
    
    @classmethod
    def load_from_file(cls, filename):
        """
        Load a configuration from a JSON file
        
        Parameters:
        filename (str): Path to the input file
        
        Returns:
        VectorConfig: Configuration object
        """
        if not os.path.exists(filename):
            print(f"Warning: Config file {filename} not found. Using default configuration.")
            return cls()
        
        with open(filename, 'r') as f:
            config_dict = json.load(f)
        
        return cls.from_dict(config_dict)

# Default configuration
default_config = VectorConfig()
\end{lstlisting}

\subsection{visualization.py}

\begin{lstlisting}[language=Python]
#!/usr/bin/env python3
import numpy as np
import matplotlib.pyplot as plt
from mpl_toolkits.mplot3d import Axes3D

def plot_vectors_3d(R_0, R_1, R_2, R_3, figsize=(10, 8), show_legend=True):
    """
    Plot the vectors in 3D
    
    Parameters:
    R_0 (numpy.ndarray): The origin vector
    R_1, R_2, R_3 (numpy.ndarray): The three orthogonal vectors
    figsize (tuple): Figure size (width, height) in inches
    show_legend (bool): Whether to show the legend
    
    Returns:
    tuple: (fig, ax) matplotlib figure and axis objects
    """
    fig = plt.figure(figsize=figsize)
    ax = fig.add_subplot(111, projection='3d')
    
    # Plot the origin
    ax.scatter(R_0[0], R_0[1], R_0[2], color='black', s=100, label='R_0')
    
    # Plot the vectors as arrows from the origin
    vectors = [R_1, R_2, R_3]
    colors = ['r', 'g', 'b']
    labels = ['R_1', 'R_2', 'R_3']
    
    for i, (vector, color, label) in enumerate(zip(vectors, colors, labels)):
        ax.quiver(R_0[0], R_0[1], R_0[2], 
                 vector[0]-R_0[0], vector[1]-R_0[1], vector[2]-R_0[2], 
                 color=color, label=label, arrow_length_ratio=0.1)
    
    # Set labels and title
    ax.set_xlabel('X')
    ax.set_ylabel('Y')
    ax.set_zlabel('Z')
    ax.set_title('3D Plot of Orthogonal Vectors')
    
    # Set equal aspect ratio
    max_range = np.array([
        np.max([R_0[0], R_1[0], R_2[0], R_3[0]]) - np.min([R_0[0], R_1[0], R_2[0], R_3[0]]),
        np.max([R_0[1], R_1[1], R_2[1], R_3[1]]) - np.min([R_0[1], R_1[1], R_2[1], R_3[1]]),
        np.max([R_0[2], R_1[2], R_2[2], R_3[2]]) - np.min([R_0[2], R_1[2], R_2[2], R_3[2]])
    ]).max() / 2.0
    
    mid_x = (np.max([R_0[0], R_1[0], R_2[0], R_3[0]]) + np.min([R_0[0], R_1[0], R_2[0], R_3[0]])) / 2
    mid_y = (np.max([R_0[1], R_1[1], R_2[1], R_3[1]]) + np.min([R_0[1], R_1[1], R_2[1], R_3[1]])) / 2
    mid_z = (np.max([R_0[2], R_1[2], R_2[2], R_3[2]]) + np.min([R_0[2], R_1[2], R_2[2], R_3[2]])) / 2
    
    ax.set_xlim(mid_x - max_range, mid_x + max_range)
    ax.set_ylim(mid_y - max_range, mid_y + max_range)
    ax.set_zlim(mid_z - max_range, mid_z + max_range)
    
    if show_legend:
        ax.legend()
    
    return fig, ax

# Note: This is a partial listing. The full visualization.py file contains additional functions
# such as plot_vectors_2d_projection and plot_all_projections that are omitted here for brevity.
\end{lstlisting}

\subsection{\_\_init\_\_.py}

\begin{lstlisting}[language=Python]
# Generalized Orthogonal Vectors Generator and Visualizer
# This package provides tools for generating and visualizing orthogonal vectors

from .vector_utils import create_orthogonal_vectors, check_orthogonality
from .visualization import plot_vectors_3d, plot_vectors_2d_projection, plot_all_projections
from .config import VectorConfig, default_config

__all__ = [
    'create_orthogonal_vectors',
    'check_orthogonality',
    'plot_vectors_3d',
    'plot_vectors_2d_projection',
    'plot_all_projections',
    'VectorConfig',
    'default_config'
]

__version__ = '1.0.0'
\end{lstlisting}
